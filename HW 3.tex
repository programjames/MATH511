\documentclass[12pt,oneside]{article}
\usepackage[margin=1in]{geometry}
\usepackage{amssymb,amsmath,amsthm}
\usepackage{stmaryrd}
\usepackage{enumitem}
\usepackage{float}
\usepackage{graphicx}



\DeclareMathOperator{\lcm}{lcm}
\DeclareMathOperator{\ord}{ord}

\begin{document}
	\title{Math 371 Homework 36}
	
	\begin{flushright}
		\textsc{James Camacho}  \\
		Math 371
	\end{flushright}
	\begin{center}
		\textsf{HW 3} \\
		\textsf{Consistency, Stability, and Second-Order Convergence.}
	\end{center}

	We need to show the finite difference method $u''(x_j) \approx \Delta^2u_j =  \frac{1}{h^2}(u_{j-1} - 2u_j + u{j+1})$ is both consistent, stable, and second-order convergent.
	
	\textbf{Consistency.} We need to show the local error $\tau_j$ goes to zero. From the Taylor expansion about $x_j$ we have
	\begin{align*}
	\Delta^2u_j &= \frac{1}{h^2}\left[-2u_j + (u_j - hu'(x_j) + \frac{h^2}{2}u''(x_j) - \cdots) + (u_j + hu'(x_j) + \frac{h^2}{2}u''(x_j) + \cdots)\right]\\
	&= \frac{1}{h^2}\left[h^2u''(x_j) + \frac{h^4}{12}u^{(4)}(x_j) + \cdots\right]\\
	&= u''(x_j) + O(h^2).
	\end{align*}
	So,
	$$\tau_j = u''(x_j) - \Delta^2u_j = O(h^2)$$
	and
	$$\lim_{h\to 0} \tau_j = \lim_{h\to 0}O(h^2) = 0.$$
	This proves consistency.
	
	\textbf{Stable.} Consistency isn't enough for convergence. Along a similar vein, the functions $f_i: [0, 1)\to [0, 1)$ defined by the rule $f_i(x) = x^i$ converge everywhere to the zero function, but for any $i$ there are points greater than $0.5$. Most authors say these functions do not \emph{absolutely} converge to the zero function, even if they converge point-wise. Stability is the absolute convergence equivalent for finite difference methods (and consistency is the weaker condition of just convergence).
	
	We need to show that some norm over all the points in our mesh is converging to zero, not just the norm for each individual point. Typically the $\infty$-norm or 2-norm is used. We'll use the 2-norm.
	
	If $A$ is our finite difference matrix (a Toeplitz matrix with diagonals $1, -2, 1$), then we have
	$$Au = f$$
	(where $u = \langle u_1\ u_2\ \dots\ u_n\rangle$, and $f = \langle f(x_1)\ f(x_2)\ \dots\ f(x_n)\rangle$). Our true solution vector is $v = \langle u(x_1)\ u(x_2)\ \dots\ u(x_n)\rangle$. By definition,
	$$\tau = Av - f = Av - Au$$
	$$\Longleftrightarrow$$
	$$e := v-u = A^{-1}\tau$$
	where we've defined $e$ to be the error vector. Our global error is the 2-norm of $e$, or
	\begin{align*}
	\lVert e\rVert_2 &= \lVert A^{-1}\tau\rVert_2\\
	&\le \lVert A^{-1}\rVert_2 \lVert \tau\rVert_2\\
	&= \lVert A^{-1}\rVert_2 O(h^2).
	\end{align*}
	from the definition of the 2-norm of a matrix. It's a well-known fact that the 2-norm of a matrix is equal to the largest absolute value of an eigenvalue. Also, the eigenvalues of $A^{-1}$ are the reciprocals of the eigenvalues of $A$. So, let's compute the eigenvalues of $A$.
	
	First, we try adding a $1$ in each of the corners of $A$ so that
	$$A = 2I + R + R^*$$
	where $R$ is the $n\times n$ right shift rotation matrix
	$$R = \begin{bmatrix}0&1&0&\cdots&0\\0&0&1&\cdots&0\\0&0&0&\cdots&0\\\vdots&\vdots&\vdots&\ddots&\vdots\\1&0&0&\cdots&0\end{bmatrix}$$
	
	Then $R^* = R^{-1}$  so if $\lambda\in \text{eig}(R)$ we have $\overline{\lambda}$ is the eigenvalue of $R^{-1}$ with the eigenvector also just the conjugate. Also, as $R^n$ = I the eigenvalues are just $e^{2ki\pi/n}$ for $k=1, 2, \dots, n$. Then, the eigenvalues of $A$ are
	$$2 + e^{2ki\pi/n}+e^{-2ki\pi/n} = 2 + 2\cos\frac{2k\pi}{n}.$$
	(I could be off by a negative here.) Similarly we find the eigenvectors are
	$$\langle \cos\frac{2k\pi}{n}, \cos\frac{4k\pi}{n}, \dots, \cos\frac{2nk\pi}{n}\rangle.$$
	Anyway, this motivates us to try a similar eigenvalue/vector on our original eigenvector. We lose a boundary condition, so we try increasing our denominator. Indeed, we find that the eigenvalues and vectors
	$$\lambda_k = 2 + 2\cos(2k\pi h),$$
	$$g_k = \langle \cos(2k\pi h)\ \cos(4k\pi h)\ \dots\ \cos(2nk\pi h) \rangle$$
	work (recall that $h = 1/(n+1)$). Notice all the eigenvalues are bounded in $(1, 3)$, so all the eigenvalues of $A^{-1}$ are bounded in $(1/3, 1)$. We can now complete the proof of stability. We have
	$$\lVert e\rVert_2\le \lVert A\rVert_2 O(h^2) \in (O(h^2)/3, O(h^2)) = O(h^2).$$
	So it is stable.
	
	\textbf{Convergence.} It is 2nd-order consistent and stable, so by definition it is 2nd-order convergent.
	
	
\end{document}